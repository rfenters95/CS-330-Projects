%%%%%%%%%%%%%%%%%%%%%%%%%%%%%%%%%%%%%%%%%
% Short Sectioned Assignment
% LaTeX Template
%%%%%%%%%%%%%%%%%%%%%%%%%%%%%%%%%%%%%%%%%

%----------------------------------------------------------------------------------------
%	PACKAGES AND OTHER DOCUMENT CONFIGURATIONS
%----------------------------------------------------------------------------------------

\documentclass[paper=a4, fontsize=11pt]{scrartcl} % A4 paper and 11pt font size

\usepackage[T1]{fontenc} % Use 8-bit encoding that has 256 glyphs
\usepackage[english]{babel} % English language/hyphenation
\usepackage{amsmath,amsfonts,amsthm} % Math packages

\usepackage{sectsty} % Allows customizing section commands
\allsectionsfont{\centering \normalfont\scshape} % Make all sections centered, the default font and small caps

\usepackage{fancyhdr} % Custom headers and footers
\pagestyle{fancyplain} % Makes all pages in the document conform to the custom headers and footers
\fancyhead{} % No page header - if you want one, create it in the same way as the footers below
\fancyfoot[L]{} % Empty left footer
\fancyfoot[C]{} % Empty center footer
\fancyfoot[R]{\thepage} % Page numbering for right footer
\renewcommand{\headrulewidth}{0pt} % Remove header underlines
\renewcommand{\footrulewidth}{0pt} % Remove footer underlines
\setlength{\headheight}{13.6pt} % Customize the height of the header

\usepackage{graphicx}


% Number equations within sections (i.e. 1.1, 1.2, 2.1, 2.2 instead of 1, 2, 3, 4)
\numberwithin{equation}{section} 
% Number figures within sections (i.e. 1.1, 1.2, 2.1, 2.2 instead of 1, 2, 3, 4)
\numberwithin{figure}{section} 
% Number tables within sections (i.e. 1.1, 1.2, 2.1, 2.2 instead of 1, 2, 3, 4)
\numberwithin{table}{section} 

 % Removes all indentation from paragraphs
\setlength\parindent{0pt}

%----------------------------------------------------------------------------------------
%	TITLE SECTION
%----------------------------------------------------------------------------------------

\newcommand{\horrule}[1]{\rule{\linewidth}{#1}} % Create horizontal rule command with 1 argument of height
\newcommand{\figurewidth}{0.813\columnwidth}

\title{
\fontfamily{lmss}\selectfont \normalsize 
\textsc{Computer Science, Francis Marion School of Business} \\ [25pt]
\horrule{1.5pt} \\[0.4cm] % Thin top horizontal rule
\huge Chapter 1, Assignment I \\ % The assignment title
\horrule{2pt} \\[0.5cm] % Thick bottom horizontal rule
}

\author{\small Professor J. S. Lewis, M.S.} % Your name

\date{}

\begin{document}
\fontfamily{lmss}\selectfont

\maketitle % Print the title

\begin{abstract}
The purpose of this assignment is to make sure you can write, compile, and
execute programs for our iRobot Create robots.
\end{abstract}

%------------------------------------------------------------------------------
%	Robot Responses
%------------------------------------------------------------------------------
\section{Robot Actions and Responses 75\%}

Write a program that responds in the following ways:
\begin{enumerate}
  \item Cycle the power button from red to green in 16 even steps. Show each
  color for 1 second.
  \item If the left bumper is pressed, light the robot's Check Robot LED while
  the bumper is depressed.
  \item if the right bumper is pressed, light the robot's Debris LED while the
  bumper is depressed.
\end{enumerate}

\section{Analysis 25\%}
%
Write a short (1.5--2 page), well-formatted document detailing the way in which
the robot is acting and responding. Be specific, such that even if you are not
the one writing the code, you know exactly how it works.

In plain English paragraphs describe the process by which each task is
accomplished. Describe any problems you encountered or anything which worked
differently than you expected.

\end{document}
